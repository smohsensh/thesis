{\bf {چکيده: }}
مسئله یادگیری از صفر
\LTRfootnote{Zero-Shot Learning}
به دنبال پیش‌بینی دسته‌هایی در زمان آزمون است که در زمان آموزش هیچ داده‌ای از آن‌ها مشاهده نشده است و شناسایی آن‌ها با اطلاعات جانبی صورت می‌گیرد. در یک مسئله دسته‌بندی تصاویر، یادگیری بدون برد به این صورت است که تعدادی تصویر به همراه برچسب و اطلاعات جانبی به الگوریتم داده می‌شود، در زمان آزمون  اطلاعات جانبی مربوط به دسته‌های جدید و تصاویری بدون برچسب وجود دارد و هدف برچسب‌گذاری تصاویر با دسته‌های جدیدی است که اطلاعات جانبی آن‌ها داده شده. ویژگی‌های بصری و متونی که ویژگی‌های یک دسته را شرح می‌دهند، مثال‌هایی از اطلاعات جانبی مورد استفاده در این نوع مسائل هستند. 
در این گزارش حالت‌های مختلف تعریف مسئله یادگیری از صفر معرفی می‌شود. سپس کارهای پیشین انجام شده مورد بررسی قرار می‌گیرد. در ادامه یک راه‌حل پیشنهادی با استفاده از تعریف فضای نهان بر اساس کلاس‌های دیده شده در زمان آموزش معرفی می‌کنیم. در پایان نتایج روش ارائه شده با نتایج روش‌های پایه مقایسه خواهد شد.

{\bf  { واژه‌های کلیدی: }}
یادگیری از صفر، یادگیری بازنمایی، شبکه‌های عمیق

