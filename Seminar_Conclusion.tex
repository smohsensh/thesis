\section{جمع‌بندی}\label{conclusion}
در این گزارش، سیستمی برای دسته‌بندی جویبار داده ارائه کردیم که در آن، برای کاهش هزینه مربوط به برچسب زدن داده‌ها، زیرمجموعه‌ای از داده‌ها را با استفاده از روش انتخاب فعال برای برچسب‌زنی انتخاب می‌کرد و برای برچسب‌زنی آن‌ها، به جای استفاده از افراد متخصص، از جمع‌سپاری استفاده می‌کند. در بخش \ref{Intro}، به بررسی ساختار این سیستم و اجزای آن پرداختیم و آن را به سه بخش اصلی سیستم انتخاب فعال، سیستم دسته‌بندی جویبار داده و بخش تجمیع نظرات افراد غیرمتخصص تقسیم کردیم و چالش‌های موجود در هر قسمت را بیان کردیم. در بخش \ref{sec:PriorWorks}، به بررسی روش‌های ارائه شده در هر یک از این حوزه‌ها پرداختیم و دلایل قوت و ضعف هر یک را مورد بررسی قرار دادیم. در قسمت \ref{my_work}، با تمرکز بر روی بخش دسته‌بند نظارتی جویبار داده، چارچوبی کاملا نظام‌مند و احتمالاتی برای دسته‌بندی جویبار داده ارائه کردیم که با مدل کردن مسئله تغییر مفهوم به عنوان یک مسئله خوشه‌بندی پویا و استفاده از مدل‌های ناپارامتری بیزی به یک روش کارا برای دسته‌بندی دست یافتیم. سپس با مقایسه روش‌های پایه و روش پیشنهادی، موفقیت آن را بر روی یک مجموعه داده واقعی نشان دادیم. نقاط قوت و ضعف برخی روش‌های ارائه‌شده در حوزه دسته‌بندی جویبار داده نیز در جدول \ref{tab:PriorWorks}، به اختصار شرح داده شده است.

\begin{table}[h!]
 \caption{ مقایسه روش‌های ارائه شده در حوزه دسته‌بندی جویبار داده}\label{tab:PriorWorks}
 \begin{center}
\begin{tabular}{|p{4cm}|c|p{7cm}|p{7cm}|}
\hline
\rl{نام روش}&\rl{سال ارائه}& \rl{مزایا و معایب}\\
\hline
\footnotesize \rl{وزن‌دهی به درستنمایی} \cite{unbiasedstream} & 
\footnotesize 2011 &
\footnotesize \rl{+مدلی احتمالاتی و نظام‌مند} \newline \rl{- استفاده از تنها یک مدل} \newline 
 \rl{- عدم قابلیت یادگیری مفاهیم تکرارشونده} \newline \rl{- سرعت پایین در بازیابی دقت پس از تغییر مفهوم}\\

\hline

\footnotesize \rl{ترکیب دسته‌بندهای پایه} \cite{NSE} & 
\footnotesize 2011 &
\footnotesize \rl{+غنی کردن فضای فرضیه با ترکیب مدل‌های ساده} \newline \rl{+ پشتیبانی از مفاهیم تکرارشونده} \newline 
 \rl{- قوانین به‌روزرسانی مکاشفه‌ای} \newline \rl{- تعداد زیاد دسته‌بندهای پایه}\newline \rl{- نداشتن مکانیسمی برای محدود کردن تعداد دسته‌بندها}\\

\hline


\footnotesize \rl{خوشه‌بندی بردار ویژگی استخراج شده از دسته‌ها} \cite{RecurringContext} & 
\footnotesize 2010 &
\footnotesize \rl{+پشتیبانی از مفاهیم تکرارشونده} \newline \rl{- نداشتن مکانیسمی برای محدود کردن تعداد دسته‌بندها} \newline 
 \rl{- حساسیت زیاد به پارامترها}\newline \rl{- یکسان فرض کردن مفهوم تمامی داده‌های یک دسته}\\

\hline
\footnotesize \rl{روش گروهی مبتنی بر دقت} \cite{JavadEvolving} & 
\footnotesize 2013 &
\footnotesize \rl{+پشتیبانی از مفاهیم تکراشونده} \newline \rl{+روشی احتمالاتی برای انتخاب دسته‌بندی که باید به‌روز شود} \newline \rl{- روشی مکاشفه‌ای برای انتخاب مفهوم یک داده} \newline \rl{- یکسان فرض کردن مفهوم تمامی داده‌های یک دسته}\\

\hline
\footnotesize \rl{روش دسته‌بندی مبتنی بر مدل مخلوط فرآیند دیریکله} \cite{NonlinearUsingDP} & 
\footnotesize 2009 &
\footnotesize \rl{+ارائه مدلی غیرخطی بر مبنای ترکیب دسته‌بندهای ساده خطی} \newline \rl{+ تعیین تعداد دسته‌بندهای مورد نیاز بر حسب پیچیدگی مدل} \newline \rl{- عدم پشتیبانی از جویبار داده}\\

\hline


\end{tabular}
\end{center}
\end{table}
