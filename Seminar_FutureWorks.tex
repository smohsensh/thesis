\section{کارهای آتی}\label{future_works}
در ادامه مسیر تحقیق، ایده‌ی مطرح شده برای بدست آوردن نگاشت تصاویر در فضای میانی را تکمیل می‌کنیم همچنین امکان استفاده از بردارهای ویژگی برای هر یک از نمونه‌های آموزش یا اینکه وجود چندین توصیف برای یک دسته آزمون را در نظر گرفته و آن را به روش‌های نگاشت توصیف‌ها به فضای میانی وارد خواهیم کرد. با توجه به اینکه متون توصیف‌هایی با قابلیت دسترسی بیشتر نسبت به بردارهای ویژگی هستند استخراج نمایش برداری مفید در یادگیری از صفر از روی متون را مورد بررسی قرار خواهیم داد. خلاصه‌ای از مراحل و میزان پیشرفت پروژه در جدول \ref{tab:Timing} آمده است. 
 \begin{table}[h!]
 \caption{جدول زمان‌بندی\label{tab:Timing}}
 \begin{center}
\begin{tabular}{|r|c|c|c|}
\hline
\rl{عنوان فعالیت}&\rl{مدت زمان لازم}&\rl{درصد پیشرفت}&\rl{زمان اتمام}\\ \hline \hline
\rl{مطالعه و بررسی روش‌های موجود و راه‌کارهای قابل استفاده  }&\rl{3 ماه}&100&\rl{شهریور ۹۴}\\ \hline
\rl{آزمایش روش‌های موجود بر روی مجموعه داده‌های معرفی شده در مقالات و مقایسه آن‌ها}& \rl{۲ ماه}&100&\rl{آبان  ۹۴}\\ \hline
\rl{بررسی و یافتن کاستی‌های روش‌های موجود}&\rl{۱ ماه}&80&\rl{آبان ۹۴}\\ \hline
\rl{ پیشنهاد و پیاده‌سازی و ارزیابی روش جدید}&\rl{۴ ماه}& 60&\rl{اسفند ۹۴}\\ \hline
\rl{ارزیابی روش نهایی و مقایسه با روش‌های دیگر}&\rl{۲ ماه}&0&\rl{اردیبهشت ۹۵}\\ \hline
\rl{نگارش پایان‌نامه}&\rl{۲ ماه}&0&\rl{تیر ۹۵}\\ \hline
\end{tabular}
\end{center}
\end{table}
