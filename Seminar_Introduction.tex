\section{مقدمه}\label{intro}

 در حوزه یادگیری ماشین مسئله استاندارد یادگیری با نظارت به صورت‌های مختلف توسعه یافته است و به کمک این روش‌ها، یادگیری ماشین از عهده‌ی کارهای بسیار چالش‌برانگیزتری بر آمده است. بر خلاف پارادایم سنتی یادگیری با نظارت، که فرض می‌کند داده‌های فراوانی از تمام دسته‌ها برای آموزش در اختیار قرار دارد، عموم این روش‌ها به دنبال کم کردن نیاز به داده‌های برچسب‌دار در زمان آموزش هستند. 
\emph{ یادگیری نیمه‌نظارتی} \LTRfootnote{Semi-supervised learning} \cite{chapel06}
برای استفاده کردن از حجم زیاد داده‌های بدون برچسب موجود در جریان آموزش پیشنهاد شده است. 
\emph{یادگیری از تک نمونه} \LTRfootnote{One-shot learning} \cite{miller12}
سعی می‌کند یک دسته را تنها بوسیله یک نمونه‌ی برچسب‌دار از آن و البته با کمک نمونه‌های برچسب‌دار از سایر دسته‌ها شناسایی کند. 
\emph{ انتقال یادگیری} \LTRfootnote{Transfer Learning} \cite{pan10survey}
سعی می‌کند دانش به دست آمده از داده‌های یک دامنه یا برای انجام یک وظیفه را به داده‌های دامنه‌ی دیگر یا وظیفه‌ی دیگری روی داده‌ها منتقل کند. 
هیچ‌کدارم از این روش‌ها یاز به داده‌های برچسب‌دار را برای دسته‌هایی که مایل به تشخیص آن هستیم را به طور کامل از بین نمی‌برد. برای دست‌یابی به چنین هدفی، 
مسئله یادگیری از صفر را صورت‌بندی شده است \cite{bengio08}. در این مسئله در حالی که داده‌های آموزش برای بعضی از دسته‌ها هیچ نمونه‌ای در بر ندارد، به دنبال یافتن یک دسته‌بند برای آن‌ها هستیم. برای این که چنین کاری ممکن باشد فرض می‌شود که یک \emph{ توصیف} از تمامی کلاس‌ها موجود است. نیاز به حل  چنین مسئله‌ای به خصوص وقتی که تعداد دسته‌ها بسیار زیاد است رخ می‌دهد. برای مثال در بینایی ماشین تعداد دسته‌ها برابر انواع اشیای موجود در جهان است و جمع‌آوری داده‌های آموزش برای همه اگر غیر ممکن نباشد به هزینه و زمان زیادی احتیاج دارد. همانطور که در 
\cite{sala11}
نشان داده‌شده، تعداد نمونه‌های موجود برای هر دسته از قانون Zipf پیروی می‌کند و نمونه‌های فراوان برای آموزش مستقیم دسته‌بند برای همه‌ی دسته‌ها وجود ندارد. 
 یک مثال دیگر رمزگشایی فعالیت ذهنی فرد است 
\cite{hinton09}؛
یعنی تشخیص کلمه‌ای که فرد در مورد آن فکر یا صحبت می‌کنند بر اساس تصویری که از فعالیت مغزی او تهیه شده است. طبیعتاً در این مسئله تهیه تصویر یا سیگنال فعالیت مغزی برای تمامی کلمات لغت‌نامه ممکن نیست. یک موقعیت دیگر که توصیف مسئله یادگیری از صفر بر آن منطبق است دسته‌بندی دسته‌های جدید است، مانند تشخیص مدل‌های جدید محصولاتی چون خودروها که بعضی دسته‌ها در زمان آموزش اصولا وجود نداشته است. یادگیری از صفر نیز مانند بسیاری از مسائل در یادگیری ماشین با توانایی‌های یادگیری در انسان ارتباط دارد و الهام از یادگیری انسان‌ها در شکل‌گیری‌اش بی‌تاثیر نبوده است. برای مثال انسان قادر است بعد از شنیدن توصیف «حیوانی مشابه اسب با راه‌راه‌های سیاه و سفید» یک گورخر را تشخیص دهد. یا تصویر یک اسکوتر را با توصیف «وسیله‌ای دو چرخ، یک کفی صاف برای ایستادن، یک میله صلیبی شکل با دو دستگیره» تطبیق خواهد داد. 

در این نوشتار بر مسئله دسته‌بندی تصاویر از صفر تمرکز می‌کنیم؛ به این معنی که داده‌هایی که مایل به دسته‌بندی آن هستیم تصاویر هستند. در نتیجه در زمان آموزش تعدادی تصویر به همراه برچسب آن‌ها موجود است. برچسب‌هایی که در زمان آموزش وجود دارند را {\emph دسته‌های دیده شده} یا \emph{ دسته‌های آموزش} می‌نامیم. همچنین یک نوع اطلاع جانبی هر یک از دسته‌های آموزش را وصف می‌کند؛ به این اطلاعات جانبی \emph{ توصیف}  می‌گوییم. در زمان آزمون تصاویری ارائه می‌شود که به دسته‌هایی غیر از دسته‌های آموزش تعلق دارند. به این دسته‌ها با نام\emph{  دسته‌های آزمون}  یا \emph{ دسته‌های دیده‌نشده}  اشاره می‌کنیم. همچنین اطلاعات جانبی مربوط به این کلاس‌ها نیز در اختیار قرار می‌گیرد. در برخی روش‌ها فرض می‌شود توصیف دسته‌های آزمون هم در زمان آموزش قابل دسترسی است. توصیف‌ها ممکن است به صورت یک بردار از ویژگی‌های بصری \cite{farhadi09}،
 عبارات زبان طبیعی 
 \cite{ng13, mohamed13, noroz14}
 و یا دسته‌بندهای یادگرفته شده  \cite{Yu2013} باشند. بردار ویژگی مرسوم‌ترین شکل توصیف کلاس است. ویژگی‌ها با توجه به نوع مسئله و گستردگی دسته‌ها تعیین می‌شوند. اکثر ویژگی‌ها، ویژگی‌های بصری هستند مانند شکل (مانند گرد یا مستطیلی)، جنس (مانند چوبی یا فلزی) و عناصر موجود در تصویر (مانند چشم، مو، پدال و نوشته). برخی ویژگی‌ها هم ممکن است مستقیما در تصویر قابل مشاهده نباشند برای مثال در یک مجموعه دادگان که دسته‌ها انواع حیوانات هستند
 \cite{lampert09}،
 علاوه بر ویژگی‌های بصری، ویژگی‌هایی چون اهلی بودن، سریع‌ بودن یا گوشت‌خوار بودن هم وجود دارد. 
 
مباحث ادامه این گزارش به این صورت است: در بخش \ref{review} صورت‌های مختلفی مسئله یادگیری از صفر را با توجه به نوع اطلاعات جانبی مورد استفاده بیان کرده و روش‌های پیشین ارائه شده برای حل آن‌ها مرور می‌کنیم. در بخش \ref{proposed} یک روش پیشنهادی بیان می‌شود و نتایج عملی آن در بخش \ref{experiments} ارائه و روش‌های دیگر مقایسه می‌شود. بخش \ref{conclusion} به کارهای آتی، جدول زمان‌بندی پژوهش و جمع‌بندی اختصاص دارد. 


