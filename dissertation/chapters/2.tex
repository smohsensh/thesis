
%------chapter 2: literature review
\chapter{روش‌های پیشین}

استفاده از اشتراک و تمایز برخی ویژگی‌ها میان دسته‌های مختلف در بینایی ماشین مورد بررسی قرار گرفته است 
\cite{BakkerH03, TsochantaridisJHA05, ulman2005}
اما این روش‌ها به شناسایی دسته‌های کاملا جدید کمتر توجه نشان داده‌اند.

مسئله‌ی یادگیری تک‌ضرب 
\LTRfootnote{One-shot Learning}
یک مسئله نزدیک به یادگیری صفر است که پیش معرفی مسئله یادگیری از صفر مورد بررسی بوده است
\cite{miller12}.
در حقیقت می‌توان یادگیری تک‌ضرب را حالت خاصی از یادگیری از صفر در نظر گرفت که در آن توصیف دسته‌های دیده نشده به صورت یک نمونه از آن دسته ارائه شده است
\cite{bengio08}.

 پدیده شروع سرد
\LTRfootnote{cold start}
 در سامانه‌های توصیه‌گر
\LTRfootnote{Recommender Systems}
 را نیز می‌توان از حالت‌های خاص یادگیری بدون برد در نظر گرفت که در آن برای یک کاربر یا مورد جدید پیشنهاد صورت می‌گیرد.


بیان مسئله  یادگیری بدون برد به طور رسمی برای اولین بار در 
\cite{bengio08}
صورت گرفت. در آن‌جا دو رویکرد کلی برای حل مسئله یادگیری از صفر بیان می‌شود. یک روش که رویکرد فضای ورودی 
\LTRfootnote{input space view}
نامیده می‌شود، سعی در مدل کردن نگاشتی با دو ورودی دارد. یکی نمونه‌ها و دیگری توصیف دسته‌ها. این نگاشت برای نمونه‌ها و توصیف‌های مربوط به یک دسته امتیاز بالا و برای نمونه‌ها و توصیفاتی که متعلق به دسته‌ی یکسانی نیستند مقادیر کوچکی تولید می‌کند. با تخمین زدن چنین نگاشتی روی داده‌های آموزش، دسته‌بندی نمونه‌های آزمون در دسته‌هایی که تا کنون نمونه‌ای نداشته‌اند ممکن خواهد شد. به این صورت که هر نمونه با توصیف دسته‌های مختلف به این تابع داده شده و متعلق به دسته‌ای که امتیاز بیشتری بگیرد، پیش‌بینی خواهد شد. 
در روش دیگر که رویکرد فضای مدل 
\LTRfootnote{model space view}
نام دارد، مدل مربوط به هر دسته (برای مثال پارامترهای دسته‌بند مربوط به آن)، به عنوان تابعی از توصیف آن دسته در نظر گرفته می‌شود. 
