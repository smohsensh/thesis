\chapter{جمع‌بندی} \label{chap:conclusion}
\section{جمع‌بندی}
در این پژوهش مسئله یادگیری بدون برد را برای دسته‌بندی تصاویر مورد بررسی قرار دادیم. در این مسئله برخی دسته‌ها در زمان آموزش نمونه‌ای ندارند و با استفاده از یک نوع اطلاعات جانبی شناسایی می‌شوند و برای آن‌ها دسته‌بند ساخته می‌شود. ابتدا یک چهارچوب کلی برای روش‌های موجود در مسئله یادگیری بدون برد ارائه کردیم. این چهارچوب شامل سه کار ۱) نگاشت تصاویر به یک فضای میانی، ۲) نگاشت توصیف‌ها به فضای میانی و ۳) دسته‌بندی در فضای میانی بود. سپس روش‌های پیشین در قالب این چهارچوب مرور شدند. در این مرور مشاهده کردیم که به استفاده از اطلاعات بدون نظارت موجود در ساختار فضای تصاویر کمتر توجه شده است. 

در ادامه برای استفاده از اطلاعات موجود در ساختار فضای تصاویر، یک تابع مطابقت مبتنی بر خوشه‌بندی تصاویر بیان کردیم که قابلیت اضافه شدن به روش‌های پیشین و بهبود آن‌ها را داراست. با توجه به تکیه‌ی این تابع مطابقت به یک خوشه‌بندی از تصاویر یک روش خوشه‌بندی نیمه‌نظارتی ارائه دادیم که با ساختار و فرض‌های مسئله یادگیری بدون برد منطبق باشد. 
در ادامه با ترکیب تابع مطابقت و خوشه‌بندی نیمه‌نظارتی معرفی شده، یک روش برای مسئله یادگیری بدون برد پیشنهاد کردیم که به نتایجی بهتر از نتایج پیشگام روش‌های پیشین در اکثر آزمایشات دست پیدا کرد. برای رفع نقایص این روش و  افزایش بیشتر دقت دسته‌بندی، روش پیشنهادی دوم را تحت عنوان یادگیری نگاشت و خوشه‌بندی توام ارائه کردیم که محدودیت‌های ناشی از جدا بودن این مراحل در روش قبلی را بر طرف کرده و دقت دسته‌بندی را افزایش داد. 
\section{کار‌های آینده}

