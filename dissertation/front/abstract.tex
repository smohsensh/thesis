
% -------------------------------------------------------
%  Abstract
% -------------------------------------------------------


\pagestyle{empty}

%\شروع{وسط‌چین}
\مهم{چکیده}
در برخی از مسائل دسته‌بندی، ممکن است داده‌ی برچسب‌دار برای تمامی دسته‌های موجود در مسئله در دسترس نباشد. برای حل چنین مسائلی، یادگیری بدون برد از اطلاعات جانبی توصیف کننده‌ی دسته‌ها استفاده می‌کند تا برای آن‌ها دسته‌بند بسازد. به طور خاص در مسئله دسته‌بندی تصاویر زمانی که دسته‌بندی دسته‌های ریزدانه یا نوظور مطرح باشد، جمع‌آوری نمونه برای تمام دسته‌ها امکان‌پذیر نخواهد بود. در این حالت از بردارهای ویژگی یا متون و یا کلمات توصیف‌کننده‌ی دسته‌ها برای دست‌یافتن به دسته‌بند برای آنها استفاده می‌شود. در این پژوهش ما روش‌هایی ارائه می‌کنیم که علاوه بر این اطلاعات، از اطلاعات بدون نظارت موجود در ساختار فضای تصاویر نیز برای استفاده کند. با استفاده از این اطلاعات یک نگاشت خطی از فضای توصیف‌ها با فضای تصاویر پیدا می‌کنیم، به گونه‌ای که هر توصیف مربوط به دسته‌های آموزش به مرکز نمونه‌های دسته‌ی مربوط به خود نگاشته شود و توصیف مربوط به دسته‌های آزمون به نزدیکی خوشه‌ای از نمونه‌های آزمون. نشان داده خواهد شد که این روش،
می‌تواند مشکل جابجایی دامنه که باعث تضعیف عملکرد روش‌های یادگیری بدون برد می‌شود را رفع کند. کارایی روش پیشنهادی با آزمایشات عملی بر روی چهار مجموعه دادگان مرسوم برای مسئله یادگیری بدون برد سنجیده می‌شود که در سه مورد از این چهار مجموعه عملکرد بهتری نسبت به روش‌های پیشین پیشگام دارد.
%\پایان{وسط‌چین}


\پرش‌بلند
\بدون‌تورفتگی \مهم{کلیدواژه‌ها}: 
یادگیری بدون برد، انتقال یادگیری، یادگیری نیمه‌نظارتی، شبکه‌های عمیق
\صفحه‌جدید
