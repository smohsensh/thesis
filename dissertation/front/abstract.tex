
% -------------------------------------------------------
%  Abstract
% -------------------------------------------------------


\pagestyle{empty}

%\شروع{وسط‌چین}
\subsection*{چکیده}

در برخی از مسائل دسته‌بندی، ممکن است داده‌ی برچسب‌دار برای تمامی دسته‌های موجود در مسئله در دسترس نباشد. برای حل چنین مسائلی، یادگیری بدون نمود نمونه‌ای از اطلاعات جانبی توصیف کننده‌ی دسته‌ها استفاده می‌کند تا برای آن‌ها دسته‌بند بسازد. به طور خاص در مسئله دسته‌بندی بدون نمود نمونه‌ای تصاویر، زمانی که دسته‌بندی دسته‌های نوظهور یا دسته‌های بسیار شبیه به هم مطرح باشد، جمع‌آوری نمونه برای تمام دسته‌ها امکان‌پذیر نخواهد بود. در این حالت از بردارهای ویژگی یا متون و کلمات توصیف‌کننده‌ی دسته‌ها برای ساختن دسته‌بند برای آنها استفاده می‌شود. در این پژوهش، روش‌هایی ارائه می‌کنیم که علاوه بر این اطلاعات، از اطلاعات بدون نظارت موجود در ساختار فضای تصاویر نیز برای دسته‌بندی تصاویر استفاده کند. با توجه به موفقیت‌های اخیر شبکه‌های عصبی ژرف در زمینه‌ی بینایی ماشین، یک نمایش غنی از تصاویر با استفاده از این شبکه‌ها قابل بدست آوردن است. این نمایش حاوی اطلاعات بدون نظارتی است که قابلیت جداسازی نمونه‌های دسته‌های متفاوت را دارد. در بعضی از روش‌های پیشنهادی از این اطلاعات برای بهبود یادگیری نگاشت از تصاویر به یک فضای میانی، که ممکن است فضای توصیف دسته‌ها یا فضای هیستوگرام‌هایی از دسته‌های دیده شده باشد،  با شبکه‌های ژرف بهره می‌بریم. در یک روش پیشنهادی دیگر، با استفاده از این اطلاعات یک نگاشت خطی از فضای توصیف‌ها به فضای تصاویر پیدا می‌کنیم، به گونه‌ای که هر توصیف مربوط به دسته‌های آموزش به مرکز نمونه‌های دسته‌ی مربوط به خود نگاشته شود و توصیف مربوط به دسته‌های آزمون به نزدیکی خوشه‌ای از نمونه‌های آزمون. نشان داده خواهد شد که این روش،
می‌تواند مشکل جابجایی دامنه که باعث تضعیف عملکرد روش‌های یادگیری بدون نمود نمونه‌ای می‌شود را کاهش دهد. کارایی روش پیشنهادی با آزمایشات عملی بر روی چهار مجموعه دادگان مرسوم برای مسئله یادگیری بدون نمود نمونه‌ای سنجیده می‌شود که در سه مورد از این چهار مجموعه، دقت دسته‌بندی را بین ۴ تا ۳۹ درصد  نسبت به روش‌های پیشگام افزایش می‌دهد.
%\پایان{وسط‌چین}


\پرش‌بلند
\بدون‌تورفتگی \مهم{کلیدواژه‌ها}: 
یادگیری بدون نمود نمونه‌ای، انتقال یادگیری، یادگیری نیمه‌نظارتی، شبکه‌های ژرف
\صفحه‌جدید
