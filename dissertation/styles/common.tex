
% -------------------------------------------------------
%  Common Styles and Formattings
% -------------------------------------------------------

\usepackage[colorlinks,linkcolor=black,citecolor=black]{hyperref}
\usepackage{graphicx,fancyhdr,enumitem}
%\usepackage{captionx}
%\usepackage{subcaptionx}
\usepackage{amssymb,amsmath}
\usepackage[mathscr]{euscript}
\usepackage{geometry}
\usepackage{longtable}
\usepackage{subcaption}
\usepackage{packages/algorithm}
\usepackage{packages/algorithmicx}
\usepackage[noend]{packages/algpseudocode}
\usepackage{tikz}
\usetikzlibrary{bayesnet}
% \usepackage[ruled]{algorithm2e}

\usepackage{xepersian}



% -------------------- Page Format --------------------


\newgeometry{top=3.7cm,bottom=3.7cm,left=2.8cm,right=3cm,headheight=25pt}

\renewcommand{\baselinestretch}{1.5}
\linespread{1.87}
\setlength{\parskip}{0.5em}

\fancyhf{}
\rhead{\leftmark}
\lhead{\thepage}


% -------------------- Fonts --------------------

\settextfont[Scale=1.2]{XB Niloofar}
\setdigitfont[Scale=1.2]{XB Niloofar}

\defpersianfont\sayeh[Scale=1.1]{XB Kayhan Pook}

\captionsetup[figure]{labelfont=it,textfont=it}
% -------------------- Environments --------------------

\newtheorem{قضیه}{قضیه‌ی}[chapter]
\newtheorem{لم}[قضیه]{لم}
\newtheorem{مشاهده}[قضیه]{مشاهده‌ی}
\newtheorem{مسئله}{مسئله‌ی}
\newtheorem{تعریف}{تعریف}


\newenvironment{اثبات}
	{\begin{trivlist}\item[\hskip\labelsep{\em اثبات.}]}
	{\leavevmode\unskip\nobreak\quad\hspace*{\fill}{\ensuremath{{\square}}}\end{trivlist}}

\newenvironment{alg}[2]
	{\begin{latin}\settextfont[Scale=1.0]{Times New Roman}
	\begin{algorithm}[t]\caption{#1}\label{algo:#2}\vspace{0.2em}\begin{algorithmic}[1]}
	{\end{algorithmic}\vspace{0.2em}\end{algorithm}\end{latin}}


% -------------------- Numberings --------------------

% Replace periods with dashes in Farsi numbers

\makeatletter
\renewcommand \thesection {\@arabic\c@section-\@arabic\c@chapter}
\renewcommand \thesubsection {\@arabic\c@subsection-\@arabic\c@section-\@arabic\c@chapter}
\renewcommand \thetable {\@arabic\c@table-\@arabic\c@chapter}
\renewcommand \thefigure {\@arabic\c@figure-\@arabic\c@chapter}
\renewcommand \theequation {\@arabic\c@equation-\@arabic\c@chapter}
\renewcommand \theقضیه {\@arabic\c@قضیه-\@arabic\c@chapter}
\renewcommand \theلم {\@arabic\c@لم-\@arabic\c@chapter}
\renewcommand \theمشاهده {\@arabic\c@مشاهده-\@arabic\c@chapter}
\makeatother


% -------------------- Titles --------------------

\renewcommand{\listfigurename}{فهرست شکل‌ها}
\renewcommand{\listtablename}{فهرست جدول‌ها}

%\renewcommand{\bibname}{منابع}


% -------------------- Commands --------------------


\newcommand{\IR}{\ensuremath{\mathbb{R}}}
\newcommand{\IZ}{\ensuremath{\mathbb{Z}}}
\newcommand{\IN}{\ensuremath{\mathbb{N}}}
\newcommand{\IQ}{\ensuremath{\mathbb{Q}}}

\newcommand{\ceil}[1]{{\left\lceil{#1}\right\rceil}}
\newcommand{\floor}[1]{{\left\lfloor{#1}\right\rfloor}}
\newcommand{\prob}[1]{{\mbox{\tt Pr}[#1]}}
\newcommand{\set}[1]{{\{ #1 \}}}

\newcommand{\lee}{\leqslant}
\newcommand{\gee}{\geqslant}
\renewcommand{\leq}{\lee}
\renewcommand{\le}{\lee}
\renewcommand{\geq}{\gee}
\renewcommand{\ge}{\gee}

\newcommand{\کج}{\emph}
\newcommand{\مهم}{\textbf}

\newcommand{\زیرنوشت}[1]{\footnote{\lr{#1}}}
\renewcommand{\برچسب}{\label}

\newcommand{\REM}[1]{}
\renewcommand{\حذف}{\REM}

\newcommand{\cpx}[1]{\mathcal{O}(#1)}
\newcommand{\sets}[2]{#1 = \{{#2}_1,{#2}_2,\ldots,{#2}_{|#1|}\}}
\newcommand{\nphard}{ اِن‌پی-سخت }
\newcommand{\apxhard}{ تقریب-سخت }
\newcommand{\ktransmitter}[1]{ #1-فرستنده }


\newcommand{\درج‌شکل}[3]
	{\begin{figure}[t] \centering \includegraphics[width=#1cm]{figures/#3}
	\caption{#2} \label{شکل:#3}
	\end{figure}}
