\section{جمع‌بندی}\label{conclusion}
در این گزارش مسئله یادگیری از صفر به همراه نسخه‌های مختف آن و یک چارچوب کلی برای مسئله یادگیری از صفر معرفی شد. سپس به معرفی روش‌های ارائه شده برای حل این مسئله پرداختیم. با توجه به جدید بودن این مسئله و اینکه اکثر روش‌هایی که مرور شد در چندماه اخیر ارائه شده‌اند، تقسیم‌بندی استانداری از روش‌ها صورت نگرفته است. در این گزارش سعی شد برای روش‌ها یک تقسیم‌بندی بر اساس انتخاب فضای میانی، نوع نگاشت‌ها به این فضا و نوع دسته‌بند مورد استفاده ارائه شود. برخی از روش‌های پیشین مطرح در جدول  \ref{tab:PriorWorks} به طور خلاصه ذکر شده‌اند. در بخش \ref{proposed} فضای دسته‌های آموزش را به عنوان فضای میانی در نظر گرفته و روشی برای تخمین وزن‌ها در این فضا ارائه دادیم. در نهایت راه‌کاری آتی و جدول زمان‌بندی ادامه‌ی کار در بخش \ref{future_works} ارائه شد.

\begin{table}[h!]
 \caption{ مقایسه مهم‌ترین روش‌های ارائه شده برای یادگیری از صفر}\label{tab:PriorWorks}
 \begin{center}
\begin{tabular}{|p{4cm}|c|r|p{7cm}|p{7cm}|}
\hline
\rl{نام روش}&\rl{سال ارائه}& \rl{نوع توصیف قابل استفاده} &\rl{مزایا و معایب}\\
\hline
\footnotesize \lr{DAP} \cite{lampert09} & 
\footnotesize 2009 &
\footnotesize بردار ویژگی &
\footnotesize \rl{+ارائه یک چارچوب نظام‌مند  } \newline \rl{+ امکان تعویض برخی قسمت‌ها مانند نوع دسته‌بند مورد استفاده} \newline 
 \rl{- مدل نکردن ارتباط میان ویژگی‌ها} \newline \rl{- در نظر گرفتن خطای دسته‌بندی در آموزش}\\

\hline

\footnotesize \lr{ESZSL} \cite{emb15} & 
\footnotesize 2015 &
\footnotesize بردار ویژگی &
\footnotesize \rl{+ درنظرگرفتن خطای دسته‌بندر در آموزش} \newline \rl{+ دارای جواب بسته و پیاده‌سازی یک خطی} \newline 
 \rl{+ سرعت آموزش و آزمون بالا} \newline \rl{- در نظر نگرفتن ارتباط بین ویژگی‌ها}\newline \rl{- محدود بودن رابطه به روابط خطی}\\

\hline


\footnotesize \lr{COSTA} \cite{costa} & 
\footnotesize 2014 &
\footnotesize برچسب‌های دیگر &
\footnotesize \rl{+عدم نیاز به توصیف کلاس تهیه شده توسط انسان} \newline \rl{+ امکان انجام یادگیری از صفر چند برچسبی} \newline 
 \rl{- تنها امکان استفاده از اطلاع جانبی قابل دسته‌بندی}\newline \rl{- عدم امکان استفاده از ویژگی‌های غیر دودویی}\\

\hline
\footnotesize \lr{SSE} \cite{sse} & 
\footnotesize 2015 &
\footnotesize بردار ویژگی &
\footnotesize \rl{+ امکان طبیعی استفاده از ویژگی‌ها با مقدار حقیقی} \newline \rl{+ ارائه یک روش عمومی برای بیان دسته‌های آزمون بر حسب دسته‌های آموزش } \newline \rl{- مسئله بهینه‌سازی نسبتا زمان‌بر} \newline \rl{- الزاما یکسان در نظر گرفتن توزیع داده‌های آموزش و آزمون}\\
\hline
\footnotesize \rl{تشخیص هم‌دسته بودن توصیف و تصویر} \cite{agnostic} & 
\footnotesize 2015 &
\footnotesize انواع مختلف &
\footnotesize \rl{+ امکان طبیعی استفاده از انواع ویژگی‌ها} \newline \rl{+ پارامترهای مستقل از تعداد دسته‌ها } \newline \rl{- استنتاج سنگین که به اجبار تخمین زده می‌شود} \\
\hline
\footnotesize \rl{یادگیری از صفر نیمه‌نظارتی با یادگیری نمایش برچسب‌ها} \cite{semi15} & 
\footnotesize 2015 &
\footnotesize بردار ویژگی یا بدون توصیف &
\footnotesize \rl{+ یادگیری نمایش برچسب‌ها طوری که متمایزکننده‌ی دسته‌ها شود} \newline \rl{+ دسته‌بندی روی تمام دسته‌های آموزش و آزمون } \newline \rl{+ امکان دسته‌بندی حتی بدون توصیف با یادگیری توصیف‌ها} %\rl{- }
 \\
\hline
\footnotesize \rl{پیش‌بینی دسته‌بند از متن توصیفی} \cite{ba2015} & 
\footnotesize 2015 &
\footnotesize متن &
\footnotesize \rl{+ معرفی دسته‌بند پیچشی} \newline \rl{- استخراج ویژگی‌های نه چندان خوب از متن} \newline \rl{- جمع‌آوری متون مناسب ممکن است هزینه‌بر باشد} \\
\hline
\footnotesize \rl{DeViSE} \cite{devise} & 
\footnotesize 2014 &
\footnotesize نام دسته‌ها &
\footnotesize \rl{+ عدم نیاز به تهیه توصیف توسط انسان } \newline \rl{+ بهره‌گیری از پیش‌آموزش روی داده‌های فراوان} \newline \rl{- عدم دسته‌بندی دقیق برای دسته‌های نزدیک به هم} \\
\hline

\end{tabular}
\end{center}
\end{table}
