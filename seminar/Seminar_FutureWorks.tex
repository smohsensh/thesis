\section{کارهای آتی}\label{future_works}
در ادامه مسیر تحقیق
ایده‌ی مطرح شده برای بدست آوردن نگاشت تصاویر در فضای میانی تکمیل خواهد شد تا نزدیکی نگاشت‌های بدست آمده به نگاشت‌های مربوط به توصیف‌ها در جریان یادگیری لحاظ شود. موضوع دیگری که دنبال خواهد شد استفاده از بردارهای ویژگی برای هر تصویر آموزش (در مقابل یک بردار ویژگی برای هر دسته) و وجود چندین توصیف برای یک دسته آزمون است؛ چرا که این اطلاعات برای اکثر مجموعه داده‌های واقعی در دسترس قرار دارد. علاوه بر این، به استفاده از توصیف‌های متنی و هم‌چنین اطلاعات داده‌های بدون برچسب نیز خواهیم پرداخت. دو زیر بخش آتی به کارهای آتی در این دو حوزه اختصاص دارد. 

\subsection{توصیف از نوع متن}\label{text_processing}
با توجه به اینکه متون توصیف‌هایی با قابلیت دسترسی بیشتر نسبت به بردارهای ویژگی هستند و جمع‌آوری آن‌ها از منابعی مانند دائره‌المعارف‌ها بدون هزینه و دخالت نیروی انسانی امکان‌پذیر است، این نوع توصیف گزینه‌ی بسیاری مناسب‌تری برای یادگیری از صفر در مقیاس بزرگ است. از طرفی متن‌ها دارای ساختار خاصی هستند و برخلاف بردارهای ویژگی، استفاده مستقیم از آن‌ها امکان‌پذیر نخواهد بود. یکی از کارهای آتی تلاش برای یافتن نگاشتی‌ست که ویژگی‌های بیان شده در متن را به ویژگی‌های بصری استخراج شده از تصاویر مربوط کند. با توجه به شباهت این کار به ترجمه متن به زبانی دیگر، از مدل‌های موفق ارائه شده برای ترجمه‌ی خودکار که مبتنی شبکه‌های عصبی بازگردنده 
\LTRfootnote{Recurrent}
برای این کار استفاده خواهد شد. هم‌چنین با توجه به اینکه  استخراج ویژگی از تصاویر با استفاده از شبکه‌های پیش‌آموزش دیده نتایج بسیار بهتری به دنبال دارد، برای ساخت آموزش دادن یک شبکه برای استخراج ویژگی از متن با استفاده از متون فراوان موجود روی اینترنت (مانند ویکی‌پدیا
\LTRfootnote{http://www.Wikipedia.org}
) تلاش خواهد شد. 

\subsection{یادگیری نیمه‌نظارتی}
بعضی از روش‌های اخیر 
\cite{Kodirov2015, semi15, li15max}
 این فرض که داده‌های آزمون در زمان آموزش نیز موجود هستند و تنها برچسب ندارند را اضافه کرده‌اند و البته این فرض در اکثر کاربردهای واقعی برقرار است. با این فرض امکان استفاده  از اطلاعاتی که در نمونه‌های بدون برچسب در مورد ساختار و توزیع داده‌ها وجود دارد، فراهم می‌شود. یک رویکرد برای استفاده از داده‌های آزمون، حل هم‌زمان یک مسئله دسته‌بندی روی داده‌های آموزش و یک مسئله خوشه‌بندی روی داده‌های آزمون است که در \cite{li15max} به کار گرفته شده است، اما بنظر می‌رسد تابع هزینه معرفی شده برای مدل کردن مسئله فوق نزدیکی خوشه‌ها به توصیف‌های دسته‌های آزمون را در نظر نمی‌گیرد که وارد کردن آن به تابع هزینه می‌تواند در بهبود نتایج موثر باشد. 

خلاصه‌ای از مراحل و میزان پیشرفت پروژه در جدول \ref{tab:Timing} آمده است. 
 \begin{table}[h!]
 \caption{جدول زمان‌بندی\label{tab:Timing}}
 \begin{center}
\begin{tabular}{|r|c|c|c|}
\hline
\rl{عنوان فعالیت}&\rl{مدت زمان لازم}&\rl{درصد پیشرفت}&\rl{زمان اتمام}\\ \hline \hline
\rl{مطالعه و بررسی روش‌های موجود و راه‌کارهای قابل استفاده  }&\rl{3 ماه}&100&\rl{شهریور ۹۴}\\ \hline
\rl{آزمایش روش‌های موجود بر روی مجموعه داده‌های معرفی شده در مقالات و مقایسه آن‌ها}& \rl{۲ ماه}&100&\rl{آبان  ۹۴}\\ \hline
\rl{بررسی و یافتن کاستی‌های روش‌های موجود}&\rl{۱ ماه}&60&\rl{آبان ۹۴}\\ \hline
\rl{ پیشنهاد و پیاده‌سازی و ارزیابی روش جدید}&\rl{۴ ماه}& 20&\rl{اسفند ۹۴}\\ \hline
\rl{ارزیابی روش نهایی و مقایسه با روش‌های دیگر}&\rl{۲ ماه}&0&\rl{اردیبهشت ۹۵}\\ \hline
\rl{نگارش پایان‌نامه}&\rl{۲ ماه}&0&\rl{تیر ۹۵}\\ \hline
\end{tabular}
\end{center}
\end{table}
